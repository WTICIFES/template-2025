\documentclass{wticifes}

% ------------------------------------------------
% INÍCIO DO DOCUMENTO
% ------------------------------------------------
\begin{document}

% Espaçamento 1,5 para o texto principal
\onehalfspacing

\title{TÍTULO DO DOCUMENTO}
\maketitle


\begin{resumo}
\lipsum[1]
\end{resumo}

\noindent\textbf{Palavras-chave:} key1, key2, key3

\vspace{0.5cm}
\begin{abstract}
\lipsum[1]
\end{abstract}

\noindent\textbf{Keywords:} key1, key2, key3

\section{Introdução}
Este é um modelo de documento que segue as especificações solicitadas. O texto está formatado com fonte tamanho 12, espaçamento entre linhas de 1,5 cm, sem espaçamento entre parágrafos. Exemplo de citação \cite{teste2023}.

Segundo \citeonline{fowler1997}.

\lipsum[1]

\section{Desenvolvimento}
As seções estão alinhadas à esquerda e sem numeração, conforme solicitado
nos requisitos.

\lipsum[1]

Abaixo segue um exemplo de citação longa, que deve ser formatada com recuo, espaçamento simples e fonte tamanho 10:

\begin{quote}
Esta é uma citação longa que deve ser formatada em espaço simples de entrelinhas e fonte tamanho 10. O texto está alinhado com recuo à esquerda para diferenciar do texto principal do documento, conforme as especificações solicitadas nos requisitos.
\end{quote}

\lipsum[3]

\begin{figure}
    \centering
    \includegraphics[width=0.5\linewidth]{img1.jpg}
    \caption{Descrição}
    \label{fig:enter-label}
\end{figure}

\section{Considerações Finais}
\lipsum[2]

% Referências bibliográficas
\bibliographystyle{abntex2-alf}  % Estilo autor-data da ABNT
\bibliography{ref}  % Arquivo .bib contendo as referências


\end{document}